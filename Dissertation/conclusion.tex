\chapter*{Заключение}                       % Заголовок
\addcontentsline{toc}{chapter}{Заключение}  % Добавляем его в оглавление

%% Согласно ГОСТ Р 7.0.11-2011:
%% 5.3.3 В заключении диссертации излагают итоги выполненного исследования, рекомендации, перспективы дальнейшей разработки темы.
%% 9.2.3 В заключении автореферата диссертации излагают итоги данного исследования, рекомендации и перспективы дальнейшей разработки темы.
%% Поэтому имеет смысл сделать эту часть общей и загрузить из одного файла в автореферат и в диссертацию:
Одной из важнейших задач современных трансляторов является полноценное использование ресурсов системы. Чтобы решить эту задачу современные компанию ведут совместную разработку программного и аппаратного обеспечивания. Компиляторы помогают скрывать недостатки и подчеркивать важные особенности аппаратуры. В этой работе был продемонстрирован полноценный процесс, от анализа существующих решений и методологии тестирования, до разработки оптимизаций и замеров производительности. 

Основные результаты работы заключаются в следующем:
\begin{enumerate}
	\item На основе анализа современных технологий оптимизации приложений были выдвинуты гипотезы и направления исследования для последующей их оптимизации с учетом недостатков целевой архитектуры.
	\item Предварительный анализ производительности с помощью таких приложений как \textbf{perf, radare2, GEM5} позволил доказать наличие возможности  для оптимизаций целевых приложений на исследуемой микроархитектуре. 
	\item Было создано семь дополнительных  проходов в компиляторе GCC, а также предложено 5 улучшений существующих оптимизаций.
	\item Разработанное решение позволило продемонстрировать улучшение производительности в ~11 \% на целевых тестах пакета "CPUBench int"\phantom{}  с улучшением до 74 \% на отдельных приложениях. 
	\item Разработанное решение позволило продемонстрировать улучшение производительности в ~6 \% на целевых тестах пакета "CPUBench fp"\phantom{}  с улучшением до 40 \% на отдельных приложениях. Для тестов пакета "CPUBench fp"\phantom{} улучшение производительности на  одном ядре существенно отличается и составляет ~12 \% на целевых тестах.
\end{enumerate}

В заключение хочется выразить благодарность всем коллегам и студентам, без которых данная работа не была бы возможной. Отдельная благодарность и большая признательность выражается научном руководителю Доброву А.Д за поддержку и помощь на всем научном пути автора.
