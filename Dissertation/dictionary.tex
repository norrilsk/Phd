\chapter*{Словарь терминов}             % Заголовок
\addcontentsline{toc}{chapter}{Словарь терминов}  % Добавляем его в оглавление

\textbf{AES} - Advanced Encryption Standard  - Симметричный алгоритм блочного шифрования.

\textbf{ASLR} - Address Space Layout Randomization -Технология операционной системы рандомизации размещения адресного пространства.

\textbf{ARM} - Advanced RISC Machine - Описание архитектуры компьютера, разработанной компанией  ARM Limited.

\textbf{CCL} - CPU CLaster - блок ядер центрального процессора.

\textbf{CFG} - Control Flow Graph - Граф потока управления.

\textbf{CISC} - Complex Instruction Set Computing  - Сложная система команд, имеющая произвольный размер машинной инструкции и широкий набор операторов-инструкций.

\textbf{CPU} - Central Processing Unit - Центральное Вычислительное устройство.

\textbf{DFG} - Data Flow Graph  - Граф потока данных.

\textbf{DRAM} - Dynamic Random Access Memory - Динамическая энергозависимая память произвольного доступа.

\textbf{GCC} - GNU Compiler Collection - Коллекция компиляторов языков С/C++,Fortran, GO, D, ObjC и др.

\textbf{GNU} - GNU is Not Unix - Проект по разработке открытого программного обеспечения, основанный Ричардом Столлманом в 1983 году.

\textbf{IPC} - Instruction per cycle - Количество инструкций, выполняемых за один машинный такт. 

\textbf{PMU} - Performance Monitoring Unit - Блок процессора, предназначенный для исследования отслеживания различных событий.

\textbf{RISC} - Reduced Instruction Set Computer -  Вычислитель, использующий упрощенный набор команд. Имеет фиксированный размер машинной инструкции и простые для исполнения операции. 

\textbf{SCCL} - Super CPU CLaster - Блок центрального процессора, состоящий из некоторого количества ССL, контроллер памяти и  дополнительный уровень кеширования памяти.

\textbf{SICL} Super IO Cluster - Блок центрального процессора, содержащий интерфейсы взаимодействия.

\textbf{SSA} - Static Single Assignment Form - Внутреннее представление программы компилятором, в котором каждой переменной значение присваивается единожды.





