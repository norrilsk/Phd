
{\actuality} В современном мире, где преобладают мобильные системы, роль серверных вычислений постоянно возрастает. В период с 2013 до 2023 год объем денежных средств, вкладываемых  международными компаниями в облачные вычисления увеличился в 5 раз\ifsynopsis\else\phantom{ }\cite{spendingCloud,alam2020cloud}\phantom{ }\fi и достиг сотни миллиардов долларов США. Многие современные приложения не могут быть запущены на мобильных и домашних устройствах в силу огромного количества потребляемых ресурсов\ifsynopsis\else\phantom{ }\cite{marinescu2022cloud}\fi. Сегодня цена вычислительных мощностей варьируется от 60 до 1500  долларов в месяц за единицу\ifsynopsis\else\phantom{ }\cite{costCloud}\fi.  У одной только российской компании Яндекс на сегодняшний день имеется от 150 000  до 200 000 серверов в 5 датацентрах по всему миру\ifsynopsis\else \phantom{ }\cite{Yndx2}\fi. Соответственно вопрос производительности программного обеспечения на серверах стоит достаточно остро.

Из года в год различные вендоры вычислительной техники соревнуются в производительности своей продукции. Основными компонентами производительности процессора являются IPC (количество инструкций, исполняемы за один машинный такт), частота и количество выполненных инструкций\ifsynopsis\else\phantom{ }\cite{hennessy2011computer}\fi. Со стороны компилятора можно повлиять на количество инструкций и лучше использовать возможности оборудования, чтобы увеличить IPC\ifsynopsis\else \cite{alvares2021instruction}\fi.

$$performance = IPC * \dfrac{frequency}{executed\_instructions}$$

GCC (GNU Compiler Colection) — это хорошо известный компилятор\ifsynopsis\else\phantom{ }\cite{schmitt2020performance}\fi, который содержит десятки различных архитектурно-зависимых и независимых от архитектуры оптимизаций\ifsynopsis\else\phantom{ }\cite {rodriguez2021compiler}\fi. Имеет открытый исходный код и ориентирован на оптимизацию C, C++ и Fortran.

Настройка компилятора для конкретной архитектуры или набора тестов — это необходимая работа, которая помогает компании продемонстрировать наилучшую производительность. Например, Дмитрий Мельник и др. all \ifsynopsis\else\phantom{ }\cite{melnik2010case}\phantom{ }\fi оптимизировали GCC для библиотеки растеризации libevas, продемонстрировав, что GCC имеет недостатки в алгоритме распределения регистров для ARM. Кроме того,в этой же работе был проведен анализ различных типов предварительной выборки данных. Многие другие авторы предлагали методы автоматической конфигурации или настройки компилятора \ifsynopsis\else\phantom{ }\cite{plotnikov2013automatic,ashouri2018survey,cereda2020collaborative}\fi. Однако данная работа была сосредоточена не на настройке существующих оптимизаций, а на их усовершенствовании и разработке новых.


В настоящее время большинство компаний используют SpecCPU 2017 для измерения производительности компьютера \ifsynopsis\else \cite{panda2018wait,bucek2018spec}\fi. Недавно был представлен новый инструмент оценки производительности под названием CPUBench\ifsynopsis\else\phantom{ }\cite {lu2023cpubench}\fi. В ходе исследования выяснилось, что GCC очень хорошо настроен для SpecCPU 2017, в то время как для другого набора тестов (CPUBench) все еще остается много возможностей для улучшения.

%\ifsynopsis
%Этот абзац появляется только в~автореферате.
%Для формирования блоков, которые будут обрабатываться только в~автореферате,
%аведена проверка условия \verb!\!\verb!ifsynopsis!.
%Значение условия задаётся в~основном файле документа (\verb!synopsis.tex! для
%автореферата).
%\else
%Этот абзац появляется только в~диссертации.
%Через проверку условия \verb!\!\verb!ifsynopsis!, задаваемого в~основном файле
%документа (\verb!dissertation.tex! для диссертации), можно сделать новую
%команду, обеспечивающую появление цитаты в~диссертации, но~не~в~автореферате.
%\fi

% {\progress}
% Этот раздел должен быть отдельным структурным элементом по
% ГОСТ, но он, как правило, включается в описание актуальности
% темы. Нужен он отдельным структурынм элемементом или нет ---
% смотрите другие диссертации вашего совета, скорее всего не нужен.

{\aim} данной работы является 
разработка новой версии компилятора GCC, ориентированного на микроархитектуру китайского производителя кристалла, основанного на архитектуре ARMV8-64, 
и компьютера целиком.
Для~достижения поставленной цели необходимо было решить следующие {\tasks}:
\begin{enumerate}[beginpenalty=10000] % https://tex.stackexchange.com/a/476052/104425
  \item Выяснить слабые места анализируемой микроархитектуры процессора Kunpeng c помощью моделей и экспериментов.
  \item Исследовать текущую производительность тестовых пакетов CPUbench и SpecCPU 2017, скомпилированных GCC на процессоре Kunpeng.
  \item Исследовать похожие проблемы в научной литературе и предложить решение, позволяющее увеличить производительность целевых приложений.
  \item Исследовать код целевого набора приложений на предмет неоптимальностей с точки зрения целевой микроархитектуры.
  \item Разработать продуктовое решение в компиляторе GCC, позволяющее получить ускорение на целевых тестах компании.
  \item Протестировать разработанные методы на реальных приложениях. 
\end{enumerate}

Тема и содержание диссертационной работы соответствует паспорту научной
специальности 2.3.5 – Математическое и программное обеспечение вычислительных машин, комплексов и компьютерных сетей, в частности, пунктам:

п. 1 – Модели, методы и алгоритмы проектирования и анализа программ
и программных систем, их эквивалентных преобразований, верификации и тестирования.

п. 3 – Модели, методы, алгоритмы, языки и программные инструменты
для организации взаимодействия программ и программных систем.

{\novelty}
\begin{enumerate}[beginpenalty=10000] % https://tex.stackexchange.com/a/476052/104425
  \item Выполнено оригинальное исследование производительности целевой микроархитектуры, которое показало наличие недостатков.
  \item Разработано и имплементировано в продукт более десяти различных оптимизаций в компиляторе GCC, которые помогают компенсировать найденные недостатки.
  \item Впервые была исследована проблема производительности широких инструкций доступа в память и предложена оптимизация, разрешающая ее. 
  \item Впервые был представлен алгоритм автоматического подбора вероятностей условных переходов для компилятора GCC.
\end{enumerate}

{\influence}  данной работы заключается в использовании
разработанных методов и алгоритмов в  компиляторе с открытым исходным кодом  ООО «Техкомпания Хуавэй»  openEuler GCC и его
инфраструктуре для оптимизации приложений и последующем использовании пользователями. Реализованные оптимизации, верификации
и методы получения профильной информации позволили получить прирост производительности
на целевых приложениях компании.

Теоретическая значимость диссертационной работы заключается в разработке
новых алгоритмов и методов микроархитектурных оптимизаций, позволяющих раскрыть потенциал производительности целевой архитектуры.


{\defpositions}
\begin{enumerate}[beginpenalty=10000] % https://tex.stackexchange.com/a/476052/104425
  \item Улучшение алгоритма преобразования условных переходов.
  \item Алгоритм шаблонной оптимизации двойного умножения.
  \item Алгоритм разбиения широких инструкций доступа в память.
  \item Алгоритм слияния "хвостов"\phantom{ } базовых блоков.
  \item Алгоритм векторизации "ленивых"\phantom{ } вычислений.
  \item Методология оценки вероятностей условных переходов.
\end{enumerate}

{\reliability} полученных результатов и выводов обеспечивается подробным описанием проведенных экспериментом, а также возможностью их повторения. Результаты находятся в соответствии с результатами, полученными другими авторами.


{\probation}
Основные результаты работы докладывались~на:
\begin{enumerate}[beginpenalty=10000] % https://tex.stackexchange.com/a/476052/104425
	\item 63-й всероссийской научной конференции московского физико–
	технического института (государственного университета), Москва,
	ноябрь 2020 г.
	\item 66-й всероссийской научной конференции московского физико–
	технического института (государственного университета), Москва,
	ноябрь 2024 г.
	\item 4-й Международной конференции о достижениях вычислительных технологий и искусственного интеллекта, Касабланка, 2024 г.
\end{enumerate}

{\contribution} В течение нескольких лет автор данной диссертации являлся техническим руководителем команды из 10 человек по разработке openEuler GCC в России. В течение этого времени и были разработаны представленные в данной работе алгоритмы и оптимизации. Часть из них, такие как  улучшение преобразования условных переходов или векторизация циклов с малым числом итераций  были разработаны автором полностью самостоятельно. Некоторые оптимизации, такие как векторизация "ленивых"\phantom{ } вычислений, разбиение широких инструкций доступа в память, автоматический подбор вероятностей условных переходов и др., были разработаны студентами, находящимися под непосредственным руководством автора данной диссертации. Отдельные оптимизации были разработаны инженерами, находящимися под технических руководством автора.
\ifnumequal{\value{bibliosel}}{0}
{%%% Встроенная реализация с загрузкой файла через движок bibtex8. (При желании, внутри можно использовать обычные ссылки, наподобие `\cite{vakbib1,vakbib2}`).
    {\publications} Основные результаты по теме диссертации изложены
    в~XX~печатных работах,
    X из которых изданы в журналах, рекомендованных ВАК,
    X "--- в тезисах докладов.
}%
{%%% Реализация пакетом biblatex через движок biber
    \begin{refsection}[bl-author, bl-registered]
        % Это refsection=1.
        % Процитированные здесь работы:
        %  * подсчитываются, для автоматического составления фразы "Основные результаты ..."
        %  * попадают в авторскую библиографию, при usefootcite==0 и стиле `\insertbiblioauthor` или `\insertbiblioauthorgrouped`
        %  * нумеруются там в зависимости от порядка команд `\printbibliography` в этом разделе.
        %  * при использовании `\insertbiblioauthorgrouped`, порядок команд `\printbibliography` в нём должен быть тем же (см. biblio/biblatex.tex)
        %
        % Невидимый библиографический список для подсчёта количества публикаций:
        \printbibliography[heading=nobibheading, section=1, env=countauthorvak,          keyword=biblioauthorvak]%
        \printbibliography[heading=nobibheading, section=1, env=countauthorwos,          keyword=biblioauthorwos]%
        \printbibliography[heading=nobibheading, section=1, env=countauthorscopus,       keyword=biblioauthorscopus]%
        \printbibliography[heading=nobibheading, section=1, env=countauthorconf,         keyword=biblioauthorconf]%
        \printbibliography[heading=nobibheading, section=1, env=countauthorother,        keyword=biblioauthorother]%
        \printbibliography[heading=nobibheading, section=1, env=countregistered,         keyword=biblioregistered]%
        \printbibliography[heading=nobibheading, section=1, env=countauthorpatent,       keyword=biblioauthorpatent]%
        \printbibliography[heading=nobibheading, section=1, env=countauthorprogram,      keyword=biblioauthorprogram]%
        \printbibliography[heading=nobibheading, section=1, env=countauthor,             keyword=biblioauthor]%
        \printbibliography[heading=nobibheading, section=1, env=countauthorvakscopuswos, filter=vakscopuswos]%
        \printbibliography[heading=nobibheading, section=1, env=countauthorscopuswos,    filter=scopuswos]%
        %
        \nocite{*}%
        %
        {\publications} Основные результаты по теме диссертации изложены в~\arabic{citeauthor}~печатных работах,
        \arabic{citeauthorvak} из которых изданы в журналах, рекомендованных ВАК\sloppy%
        \ifnum \value{citeauthorscopuswos}>0%
            , \arabic{citeauthorscopuswos} "--- в~периодических научных журналах, индексируемых Web of~Science и Scopus\sloppy%
        \fi%
        \ifnum \value{citeauthorconf}>0%
            , \arabic{citeauthorconf} "--- в~тезисах докладов.
        \else%
            .
        \fi%
        \ifnum \value{citeregistered}=1%
            \ifnum \value{citeauthorpatent}=1%
                Зарегистрирован \arabic{citeauthorpatent} патент.
            \fi%
            \ifnum \value{citeauthorprogram}=1%
                Зарегистрирована \arabic{citeauthorprogram} программа для ЭВМ.
            \fi%
        \fi%
        \ifnum \value{citeregistered}>1%
            Зарегистрированы\ %
            \ifnum \value{citeauthorpatent}>0%
            \formbytotal{citeauthorpatent}{патент}{}{а}{}\sloppy%
            \ifnum \value{citeauthorprogram}=0 . \else \ и~\fi%
            \fi%
            \ifnum \value{citeauthorprogram}>0%
            \formbytotal{citeauthorprogram}{программ}{а}{ы}{} для ЭВМ.
            \fi%
        \fi%
        % К публикациям, в которых излагаются основные научные результаты диссертации на соискание учёной
        % степени, в рецензируемых изданиях приравниваются патенты на изобретения, патенты (свидетельства) на
        % полезную модель, патенты на промышленный образец, патенты на селекционные достижения, свидетельства
        % на программу для электронных вычислительных машин, базу данных, топологию интегральных микросхем,
        % зарегистрированные в установленном порядке.(в ред. Постановления Правительства РФ от 21.04.2016 N 335)
    \end{refsection}%
    \begin{refsection}[bl-author, bl-registered]
        % Это refsection=2.
        % Процитированные здесь работы:
        %  * попадают в авторскую библиографию, при usefootcite==0 и стиле `\insertbiblioauthorimportant`.
        %  * ни на что не влияют в противном случае
        \nocite{vakbib2}%vak
        \nocite{patbib1}%patent
        \nocite{progbib1}%program
        \nocite{bib1}%other
        \nocite{confbib1}%conf
    \end{refsection}%
        %
        % Всё, что вне этих двух refsection, это refsection=0,
        %  * для диссертации - это нормальные ссылки, попадающие в обычную библиографию
        %  * для автореферата:
        %     * при usefootcite==0, ссылка корректно сработает только для источника из `external.bib`. Для своих работ --- напечатает "[0]" (и даже Warning не вылезет).
        %     * при usefootcite==1, ссылка сработает нормально. В авторской библиографии будут только процитированные в refsection=0 работы.
}

%При использовании пакета \verb!biblatex! будут подсчитаны все работы, добавленные
%в файл \verb!biblio/author.bib!. Для правильного подсчёта работ в~различных
%системах цитирования требуется использовать поля:
%\begin{itemize}
%        \item \texttt{authorvak} если публикация индексирована ВАК,
%        \item \texttt{authorscopus} если публикация индексирована Scopus,
%        \item \texttt{authorwos} если публикация индексирована Web of Science,
%        \item \texttt{authorconf} для докладов конференций,
%        \item \texttt{authorpatent} для патентов,
%        \item \texttt{authorprogram} для зарегистрированных программ для ЭВМ,
%        \item \texttt{authorother} для других публикаций.
%\end{itemize}
%Для подсчёта используются счётчики:
%\begin{itemize}
%        \item \texttt{citeauthorvak} для работ, индексируемых ВАК,
%        \item \texttt{citeauthorscopus} для работ, индексируемых Scopus,
%        \item \texttt{citeauthorwos} для работ, индексируемых Web of Science,
%        \item \texttt{citeauthorvakscopuswos} для работ, индексируемых одной из трёх баз,
%        \item \texttt{citeauthorscopuswos} для работ, индексируемых Scopus или Web of~Science,
%        \item \texttt{citeauthorconf} для докладов на конференциях,
%        \item \texttt{citeauthorother} для остальных работ,
%        \item \texttt{citeauthorpatent} для патентов,
%        \item \texttt{citeauthorprogram} для зарегистрированных программ для ЭВМ,
%        \item \texttt{citeauthor} для суммарного количества работ.
%\end{itemize}
% Счётчик \texttt{citeexternal} используется для подсчёта процитированных публикаций;
% \texttt{citeregistered} "--- для подсчёта суммарного количества патентов и программ для ЭВМ.

%Для добавления в список публикаций автора работ, которые не были процитированы в
%автореферате, требуется их~перечислить с использованием команды \verb!\nocite! в
%\verb!Synopsis/content.tex!.
